\documentclass[11pt]{article}

    \usepackage[breakable]{tcolorbox}
    \usepackage{parskip} % Stop auto-indenting (to mimic markdown behaviour)
    

    % Basic figure setup, for now with no caption control since it's done
    % automatically by Pandoc (which extracts ![](path) syntax from Markdown).
    \usepackage{graphicx}
    % Keep aspect ratio if custom image width or height is specified
    \setkeys{Gin}{keepaspectratio}
    % Maintain compatibility with old templates. Remove in nbconvert 6.0
    \let\Oldincludegraphics\includegraphics
    % Ensure that by default, figures have no caption (until we provide a
    % proper Figure object with a Caption API and a way to capture that
    % in the conversion process - todo).
    \usepackage{caption}
    \DeclareCaptionFormat{nocaption}{}
    \captionsetup{format=nocaption,aboveskip=0pt,belowskip=0pt}

    \usepackage{float}
    \floatplacement{figure}{H} % forces figures to be placed at the correct location
    \usepackage{xcolor} % Allow colors to be defined
    \usepackage{enumerate} % Needed for markdown enumerations to work
    \usepackage{geometry} % Used to adjust the document margins
    \usepackage{amsmath} % Equations
    \usepackage{amssymb} % Equations
    \usepackage{textcomp} % defines textquotesingle
    % Hack from http://tex.stackexchange.com/a/47451/13684:
    \AtBeginDocument{%
        \def\PYZsq{\textquotesingle}% Upright quotes in Pygmentized code
    }
    \usepackage{upquote} % Upright quotes for verbatim code
    \usepackage{eurosym} % defines \euro

    \usepackage{iftex}
    \ifPDFTeX
        \usepackage[T1]{fontenc}
        \IfFileExists{alphabeta.sty}{
              \usepackage{alphabeta}
          }{
              \usepackage[mathletters]{ucs}
              \usepackage[utf8x]{inputenc}
          }
    \else
        \usepackage{fontspec}
        \usepackage{unicode-math}
    \fi

    \usepackage{fancyvrb} % verbatim replacement that allows latex
    \usepackage{grffile} % extends the file name processing of package graphics
                         % to support a larger range
    \makeatletter % fix for old versions of grffile with XeLaTeX
    \@ifpackagelater{grffile}{2019/11/01}
    {
      % Do nothing on new versions
    }
    {
      \def\Gread@@xetex#1{%
        \IfFileExists{"\Gin@base".bb}%
        {\Gread@eps{\Gin@base.bb}}%
        {\Gread@@xetex@aux#1}%
      }
    }
    \makeatother
    \usepackage[Export]{adjustbox} % Used to constrain images to a maximum size
    \adjustboxset{max size={0.9\linewidth}{0.9\paperheight}}

    % The hyperref package gives us a pdf with properly built
    % internal navigation ('pdf bookmarks' for the table of contents,
    % internal cross-reference links, web links for URLs, etc.)
    \usepackage{hyperref}
    % The default LaTeX title has an obnoxious amount of whitespace. By default,
    % titling removes some of it. It also provides customization options.
    \usepackage{titling}
    \usepackage{longtable} % longtable support required by pandoc >1.10
    \usepackage{booktabs}  % table support for pandoc > 1.12.2
    \usepackage{array}     % table support for pandoc >= 2.11.3
    \usepackage{calc}      % table minipage width calculation for pandoc >= 2.11.1
    \usepackage[inline]{enumitem} % IRkernel/repr support (it uses the enumerate* environment)
    \usepackage[normalem]{ulem} % ulem is needed to support strikethroughs (\sout)
                                % normalem makes italics be italics, not underlines
    \usepackage{soul}      % strikethrough (\st) support for pandoc >= 3.0.0
    \usepackage{mathrsfs}
    

    
    % Colors for the hyperref package
    \definecolor{urlcolor}{rgb}{0,.145,.698}
    \definecolor{linkcolor}{rgb}{.71,0.21,0.01}
    \definecolor{citecolor}{rgb}{.12,.54,.11}

    % ANSI colors
    \definecolor{ansi-black}{HTML}{3E424D}
    \definecolor{ansi-black-intense}{HTML}{282C36}
    \definecolor{ansi-red}{HTML}{E75C58}
    \definecolor{ansi-red-intense}{HTML}{B22B31}
    \definecolor{ansi-green}{HTML}{00A250}
    \definecolor{ansi-green-intense}{HTML}{007427}
    \definecolor{ansi-yellow}{HTML}{DDB62B}
    \definecolor{ansi-yellow-intense}{HTML}{B27D12}
    \definecolor{ansi-blue}{HTML}{208FFB}
    \definecolor{ansi-blue-intense}{HTML}{0065CA}
    \definecolor{ansi-magenta}{HTML}{D160C4}
    \definecolor{ansi-magenta-intense}{HTML}{A03196}
    \definecolor{ansi-cyan}{HTML}{60C6C8}
    \definecolor{ansi-cyan-intense}{HTML}{258F8F}
    \definecolor{ansi-white}{HTML}{C5C1B4}
    \definecolor{ansi-white-intense}{HTML}{A1A6B2}
    \definecolor{ansi-default-inverse-fg}{HTML}{FFFFFF}
    \definecolor{ansi-default-inverse-bg}{HTML}{000000}

    % common color for the border for error outputs.
    \definecolor{outerrorbackground}{HTML}{FFDFDF}

    % commands and environments needed by pandoc snippets
    % extracted from the output of `pandoc -s`
    \providecommand{\tightlist}{%
      \setlength{\itemsep}{0pt}\setlength{\parskip}{0pt}}
    \DefineVerbatimEnvironment{Highlighting}{Verbatim}{commandchars=\\\{\}}
    % Add ',fontsize=\small' for more characters per line
    \newenvironment{Shaded}{}{}
    \newcommand{\KeywordTok}[1]{\textcolor[rgb]{0.00,0.44,0.13}{\textbf{{#1}}}}
    \newcommand{\DataTypeTok}[1]{\textcolor[rgb]{0.56,0.13,0.00}{{#1}}}
    \newcommand{\DecValTok}[1]{\textcolor[rgb]{0.25,0.63,0.44}{{#1}}}
    \newcommand{\BaseNTok}[1]{\textcolor[rgb]{0.25,0.63,0.44}{{#1}}}
    \newcommand{\FloatTok}[1]{\textcolor[rgb]{0.25,0.63,0.44}{{#1}}}
    \newcommand{\CharTok}[1]{\textcolor[rgb]{0.25,0.44,0.63}{{#1}}}
    \newcommand{\StringTok}[1]{\textcolor[rgb]{0.25,0.44,0.63}{{#1}}}
    \newcommand{\CommentTok}[1]{\textcolor[rgb]{0.38,0.63,0.69}{\textit{{#1}}}}
    \newcommand{\OtherTok}[1]{\textcolor[rgb]{0.00,0.44,0.13}{{#1}}}
    \newcommand{\AlertTok}[1]{\textcolor[rgb]{1.00,0.00,0.00}{\textbf{{#1}}}}
    \newcommand{\FunctionTok}[1]{\textcolor[rgb]{0.02,0.16,0.49}{{#1}}}
    \newcommand{\RegionMarkerTok}[1]{{#1}}
    \newcommand{\ErrorTok}[1]{\textcolor[rgb]{1.00,0.00,0.00}{\textbf{{#1}}}}
    \newcommand{\NormalTok}[1]{{#1}}

    % Additional commands for more recent versions of Pandoc
    \newcommand{\ConstantTok}[1]{\textcolor[rgb]{0.53,0.00,0.00}{{#1}}}
    \newcommand{\SpecialCharTok}[1]{\textcolor[rgb]{0.25,0.44,0.63}{{#1}}}
    \newcommand{\VerbatimStringTok}[1]{\textcolor[rgb]{0.25,0.44,0.63}{{#1}}}
    \newcommand{\SpecialStringTok}[1]{\textcolor[rgb]{0.73,0.40,0.53}{{#1}}}
    \newcommand{\ImportTok}[1]{{#1}}
    \newcommand{\DocumentationTok}[1]{\textcolor[rgb]{0.73,0.13,0.13}{\textit{{#1}}}}
    \newcommand{\AnnotationTok}[1]{\textcolor[rgb]{0.38,0.63,0.69}{\textbf{\textit{{#1}}}}}
    \newcommand{\CommentVarTok}[1]{\textcolor[rgb]{0.38,0.63,0.69}{\textbf{\textit{{#1}}}}}
    \newcommand{\VariableTok}[1]{\textcolor[rgb]{0.10,0.09,0.49}{{#1}}}
    \newcommand{\ControlFlowTok}[1]{\textcolor[rgb]{0.00,0.44,0.13}{\textbf{{#1}}}}
    \newcommand{\OperatorTok}[1]{\textcolor[rgb]{0.40,0.40,0.40}{{#1}}}
    \newcommand{\BuiltInTok}[1]{{#1}}
    \newcommand{\ExtensionTok}[1]{{#1}}
    \newcommand{\PreprocessorTok}[1]{\textcolor[rgb]{0.74,0.48,0.00}{{#1}}}
    \newcommand{\AttributeTok}[1]{\textcolor[rgb]{0.49,0.56,0.16}{{#1}}}
    \newcommand{\InformationTok}[1]{\textcolor[rgb]{0.38,0.63,0.69}{\textbf{\textit{{#1}}}}}
    \newcommand{\WarningTok}[1]{\textcolor[rgb]{0.38,0.63,0.69}{\textbf{\textit{{#1}}}}}
    \makeatletter
    \newsavebox\pandoc@box
    \newcommand*\pandocbounded[1]{%
      \sbox\pandoc@box{#1}%
      % scaling factors for width and height
      \Gscale@div\@tempa\textheight{\dimexpr\ht\pandoc@box+\dp\pandoc@box\relax}%
      \Gscale@div\@tempb\linewidth{\wd\pandoc@box}%
      % select the smaller of both
      \ifdim\@tempb\p@<\@tempa\p@
        \let\@tempa\@tempb
      \fi
      % scaling accordingly (\@tempa < 1)
      \ifdim\@tempa\p@<\p@
        \scalebox{\@tempa}{\usebox\pandoc@box}%
      % scaling not needed, use as it is
      \else
        \usebox{\pandoc@box}%
      \fi
    }
    \makeatother

    % Define a nice break command that doesn't care if a line doesn't already
    % exist.
    \def\br{\hspace*{\fill} \\* }
    % Math Jax compatibility definitions
    \def\gt{>}
    \def\lt{<}
    \let\Oldtex\TeX
    \let\Oldlatex\LaTeX
    \renewcommand{\TeX}{\textrm{\Oldtex}}
    \renewcommand{\LaTeX}{\textrm{\Oldlatex}}
    % Document parameters
    % Document title
    \title{analysis}
    
    
    
    
    
    
    
% Pygments definitions
\makeatletter
\def\PY@reset{\let\PY@it=\relax \let\PY@bf=\relax%
    \let\PY@ul=\relax \let\PY@tc=\relax%
    \let\PY@bc=\relax \let\PY@ff=\relax}
\def\PY@tok#1{\csname PY@tok@#1\endcsname}
\def\PY@toks#1+{\ifx\relax#1\empty\else%
    \PY@tok{#1}\expandafter\PY@toks\fi}
\def\PY@do#1{\PY@bc{\PY@tc{\PY@ul{%
    \PY@it{\PY@bf{\PY@ff{#1}}}}}}}
\def\PY#1#2{\PY@reset\PY@toks#1+\relax+\PY@do{#2}}

\@namedef{PY@tok@w}{\def\PY@tc##1{\textcolor[rgb]{0.73,0.73,0.73}{##1}}}
\@namedef{PY@tok@c}{\let\PY@it=\textit\def\PY@tc##1{\textcolor[rgb]{0.24,0.48,0.48}{##1}}}
\@namedef{PY@tok@cp}{\def\PY@tc##1{\textcolor[rgb]{0.61,0.40,0.00}{##1}}}
\@namedef{PY@tok@k}{\let\PY@bf=\textbf\def\PY@tc##1{\textcolor[rgb]{0.00,0.50,0.00}{##1}}}
\@namedef{PY@tok@kp}{\def\PY@tc##1{\textcolor[rgb]{0.00,0.50,0.00}{##1}}}
\@namedef{PY@tok@kt}{\def\PY@tc##1{\textcolor[rgb]{0.69,0.00,0.25}{##1}}}
\@namedef{PY@tok@o}{\def\PY@tc##1{\textcolor[rgb]{0.40,0.40,0.40}{##1}}}
\@namedef{PY@tok@ow}{\let\PY@bf=\textbf\def\PY@tc##1{\textcolor[rgb]{0.67,0.13,1.00}{##1}}}
\@namedef{PY@tok@nb}{\def\PY@tc##1{\textcolor[rgb]{0.00,0.50,0.00}{##1}}}
\@namedef{PY@tok@nf}{\def\PY@tc##1{\textcolor[rgb]{0.00,0.00,1.00}{##1}}}
\@namedef{PY@tok@nc}{\let\PY@bf=\textbf\def\PY@tc##1{\textcolor[rgb]{0.00,0.00,1.00}{##1}}}
\@namedef{PY@tok@nn}{\let\PY@bf=\textbf\def\PY@tc##1{\textcolor[rgb]{0.00,0.00,1.00}{##1}}}
\@namedef{PY@tok@ne}{\let\PY@bf=\textbf\def\PY@tc##1{\textcolor[rgb]{0.80,0.25,0.22}{##1}}}
\@namedef{PY@tok@nv}{\def\PY@tc##1{\textcolor[rgb]{0.10,0.09,0.49}{##1}}}
\@namedef{PY@tok@no}{\def\PY@tc##1{\textcolor[rgb]{0.53,0.00,0.00}{##1}}}
\@namedef{PY@tok@nl}{\def\PY@tc##1{\textcolor[rgb]{0.46,0.46,0.00}{##1}}}
\@namedef{PY@tok@ni}{\let\PY@bf=\textbf\def\PY@tc##1{\textcolor[rgb]{0.44,0.44,0.44}{##1}}}
\@namedef{PY@tok@na}{\def\PY@tc##1{\textcolor[rgb]{0.41,0.47,0.13}{##1}}}
\@namedef{PY@tok@nt}{\let\PY@bf=\textbf\def\PY@tc##1{\textcolor[rgb]{0.00,0.50,0.00}{##1}}}
\@namedef{PY@tok@nd}{\def\PY@tc##1{\textcolor[rgb]{0.67,0.13,1.00}{##1}}}
\@namedef{PY@tok@s}{\def\PY@tc##1{\textcolor[rgb]{0.73,0.13,0.13}{##1}}}
\@namedef{PY@tok@sd}{\let\PY@it=\textit\def\PY@tc##1{\textcolor[rgb]{0.73,0.13,0.13}{##1}}}
\@namedef{PY@tok@si}{\let\PY@bf=\textbf\def\PY@tc##1{\textcolor[rgb]{0.64,0.35,0.47}{##1}}}
\@namedef{PY@tok@se}{\let\PY@bf=\textbf\def\PY@tc##1{\textcolor[rgb]{0.67,0.36,0.12}{##1}}}
\@namedef{PY@tok@sr}{\def\PY@tc##1{\textcolor[rgb]{0.64,0.35,0.47}{##1}}}
\@namedef{PY@tok@ss}{\def\PY@tc##1{\textcolor[rgb]{0.10,0.09,0.49}{##1}}}
\@namedef{PY@tok@sx}{\def\PY@tc##1{\textcolor[rgb]{0.00,0.50,0.00}{##1}}}
\@namedef{PY@tok@m}{\def\PY@tc##1{\textcolor[rgb]{0.40,0.40,0.40}{##1}}}
\@namedef{PY@tok@gh}{\let\PY@bf=\textbf\def\PY@tc##1{\textcolor[rgb]{0.00,0.00,0.50}{##1}}}
\@namedef{PY@tok@gu}{\let\PY@bf=\textbf\def\PY@tc##1{\textcolor[rgb]{0.50,0.00,0.50}{##1}}}
\@namedef{PY@tok@gd}{\def\PY@tc##1{\textcolor[rgb]{0.63,0.00,0.00}{##1}}}
\@namedef{PY@tok@gi}{\def\PY@tc##1{\textcolor[rgb]{0.00,0.52,0.00}{##1}}}
\@namedef{PY@tok@gr}{\def\PY@tc##1{\textcolor[rgb]{0.89,0.00,0.00}{##1}}}
\@namedef{PY@tok@ge}{\let\PY@it=\textit}
\@namedef{PY@tok@gs}{\let\PY@bf=\textbf}
\@namedef{PY@tok@ges}{\let\PY@bf=\textbf\let\PY@it=\textit}
\@namedef{PY@tok@gp}{\let\PY@bf=\textbf\def\PY@tc##1{\textcolor[rgb]{0.00,0.00,0.50}{##1}}}
\@namedef{PY@tok@go}{\def\PY@tc##1{\textcolor[rgb]{0.44,0.44,0.44}{##1}}}
\@namedef{PY@tok@gt}{\def\PY@tc##1{\textcolor[rgb]{0.00,0.27,0.87}{##1}}}
\@namedef{PY@tok@err}{\def\PY@bc##1{{\setlength{\fboxsep}{\string -\fboxrule}\fcolorbox[rgb]{1.00,0.00,0.00}{1,1,1}{\strut ##1}}}}
\@namedef{PY@tok@kc}{\let\PY@bf=\textbf\def\PY@tc##1{\textcolor[rgb]{0.00,0.50,0.00}{##1}}}
\@namedef{PY@tok@kd}{\let\PY@bf=\textbf\def\PY@tc##1{\textcolor[rgb]{0.00,0.50,0.00}{##1}}}
\@namedef{PY@tok@kn}{\let\PY@bf=\textbf\def\PY@tc##1{\textcolor[rgb]{0.00,0.50,0.00}{##1}}}
\@namedef{PY@tok@kr}{\let\PY@bf=\textbf\def\PY@tc##1{\textcolor[rgb]{0.00,0.50,0.00}{##1}}}
\@namedef{PY@tok@bp}{\def\PY@tc##1{\textcolor[rgb]{0.00,0.50,0.00}{##1}}}
\@namedef{PY@tok@fm}{\def\PY@tc##1{\textcolor[rgb]{0.00,0.00,1.00}{##1}}}
\@namedef{PY@tok@vc}{\def\PY@tc##1{\textcolor[rgb]{0.10,0.09,0.49}{##1}}}
\@namedef{PY@tok@vg}{\def\PY@tc##1{\textcolor[rgb]{0.10,0.09,0.49}{##1}}}
\@namedef{PY@tok@vi}{\def\PY@tc##1{\textcolor[rgb]{0.10,0.09,0.49}{##1}}}
\@namedef{PY@tok@vm}{\def\PY@tc##1{\textcolor[rgb]{0.10,0.09,0.49}{##1}}}
\@namedef{PY@tok@sa}{\def\PY@tc##1{\textcolor[rgb]{0.73,0.13,0.13}{##1}}}
\@namedef{PY@tok@sb}{\def\PY@tc##1{\textcolor[rgb]{0.73,0.13,0.13}{##1}}}
\@namedef{PY@tok@sc}{\def\PY@tc##1{\textcolor[rgb]{0.73,0.13,0.13}{##1}}}
\@namedef{PY@tok@dl}{\def\PY@tc##1{\textcolor[rgb]{0.73,0.13,0.13}{##1}}}
\@namedef{PY@tok@s2}{\def\PY@tc##1{\textcolor[rgb]{0.73,0.13,0.13}{##1}}}
\@namedef{PY@tok@sh}{\def\PY@tc##1{\textcolor[rgb]{0.73,0.13,0.13}{##1}}}
\@namedef{PY@tok@s1}{\def\PY@tc##1{\textcolor[rgb]{0.73,0.13,0.13}{##1}}}
\@namedef{PY@tok@mb}{\def\PY@tc##1{\textcolor[rgb]{0.40,0.40,0.40}{##1}}}
\@namedef{PY@tok@mf}{\def\PY@tc##1{\textcolor[rgb]{0.40,0.40,0.40}{##1}}}
\@namedef{PY@tok@mh}{\def\PY@tc##1{\textcolor[rgb]{0.40,0.40,0.40}{##1}}}
\@namedef{PY@tok@mi}{\def\PY@tc##1{\textcolor[rgb]{0.40,0.40,0.40}{##1}}}
\@namedef{PY@tok@il}{\def\PY@tc##1{\textcolor[rgb]{0.40,0.40,0.40}{##1}}}
\@namedef{PY@tok@mo}{\def\PY@tc##1{\textcolor[rgb]{0.40,0.40,0.40}{##1}}}
\@namedef{PY@tok@ch}{\let\PY@it=\textit\def\PY@tc##1{\textcolor[rgb]{0.24,0.48,0.48}{##1}}}
\@namedef{PY@tok@cm}{\let\PY@it=\textit\def\PY@tc##1{\textcolor[rgb]{0.24,0.48,0.48}{##1}}}
\@namedef{PY@tok@cpf}{\let\PY@it=\textit\def\PY@tc##1{\textcolor[rgb]{0.24,0.48,0.48}{##1}}}
\@namedef{PY@tok@c1}{\let\PY@it=\textit\def\PY@tc##1{\textcolor[rgb]{0.24,0.48,0.48}{##1}}}
\@namedef{PY@tok@cs}{\let\PY@it=\textit\def\PY@tc##1{\textcolor[rgb]{0.24,0.48,0.48}{##1}}}

\def\PYZbs{\char`\\}
\def\PYZus{\char`\_}
\def\PYZob{\char`\{}
\def\PYZcb{\char`\}}
\def\PYZca{\char`\^}
\def\PYZam{\char`\&}
\def\PYZlt{\char`\<}
\def\PYZgt{\char`\>}
\def\PYZsh{\char`\#}
\def\PYZpc{\char`\%}
\def\PYZdl{\char`\$}
\def\PYZhy{\char`\-}
\def\PYZsq{\char`\'}
\def\PYZdq{\char`\"}
\def\PYZti{\char`\~}
% for compatibility with earlier versions
\def\PYZat{@}
\def\PYZlb{[}
\def\PYZrb{]}
\makeatother


    % For linebreaks inside Verbatim environment from package fancyvrb.
    \makeatletter
        \newbox\Wrappedcontinuationbox
        \newbox\Wrappedvisiblespacebox
        \newcommand*\Wrappedvisiblespace {\textcolor{red}{\textvisiblespace}}
        \newcommand*\Wrappedcontinuationsymbol {\textcolor{red}{\llap{\tiny$\m@th\hookrightarrow$}}}
        \newcommand*\Wrappedcontinuationindent {3ex }
        \newcommand*\Wrappedafterbreak {\kern\Wrappedcontinuationindent\copy\Wrappedcontinuationbox}
        % Take advantage of the already applied Pygments mark-up to insert
        % potential linebreaks for TeX processing.
        %        {, <, #, %, $, ' and ": go to next line.
        %        _, }, ^, &, >, - and ~: stay at end of broken line.
        % Use of \textquotesingle for straight quote.
        \newcommand*\Wrappedbreaksatspecials {%
            \def\PYGZus{\discretionary{\char`\_}{\Wrappedafterbreak}{\char`\_}}%
            \def\PYGZob{\discretionary{}{\Wrappedafterbreak\char`\{}{\char`\{}}%
            \def\PYGZcb{\discretionary{\char`\}}{\Wrappedafterbreak}{\char`\}}}%
            \def\PYGZca{\discretionary{\char`\^}{\Wrappedafterbreak}{\char`\^}}%
            \def\PYGZam{\discretionary{\char`\&}{\Wrappedafterbreak}{\char`\&}}%
            \def\PYGZlt{\discretionary{}{\Wrappedafterbreak\char`\<}{\char`\<}}%
            \def\PYGZgt{\discretionary{\char`\>}{\Wrappedafterbreak}{\char`\>}}%
            \def\PYGZsh{\discretionary{}{\Wrappedafterbreak\char`\#}{\char`\#}}%
            \def\PYGZpc{\discretionary{}{\Wrappedafterbreak\char`\%}{\char`\%}}%
            \def\PYGZdl{\discretionary{}{\Wrappedafterbreak\char`\$}{\char`\$}}%
            \def\PYGZhy{\discretionary{\char`\-}{\Wrappedafterbreak}{\char`\-}}%
            \def\PYGZsq{\discretionary{}{\Wrappedafterbreak\textquotesingle}{\textquotesingle}}%
            \def\PYGZdq{\discretionary{}{\Wrappedafterbreak\char`\"}{\char`\"}}%
            \def\PYGZti{\discretionary{\char`\~}{\Wrappedafterbreak}{\char`\~}}%
        }
        % Some characters . , ; ? ! / are not pygmentized.
        % This macro makes them "active" and they will insert potential linebreaks
        \newcommand*\Wrappedbreaksatpunct {%
            \lccode`\~`\.\lowercase{\def~}{\discretionary{\hbox{\char`\.}}{\Wrappedafterbreak}{\hbox{\char`\.}}}%
            \lccode`\~`\,\lowercase{\def~}{\discretionary{\hbox{\char`\,}}{\Wrappedafterbreak}{\hbox{\char`\,}}}%
            \lccode`\~`\;\lowercase{\def~}{\discretionary{\hbox{\char`\;}}{\Wrappedafterbreak}{\hbox{\char`\;}}}%
            \lccode`\~`\:\lowercase{\def~}{\discretionary{\hbox{\char`\:}}{\Wrappedafterbreak}{\hbox{\char`\:}}}%
            \lccode`\~`\?\lowercase{\def~}{\discretionary{\hbox{\char`\?}}{\Wrappedafterbreak}{\hbox{\char`\?}}}%
            \lccode`\~`\!\lowercase{\def~}{\discretionary{\hbox{\char`\!}}{\Wrappedafterbreak}{\hbox{\char`\!}}}%
            \lccode`\~`\/\lowercase{\def~}{\discretionary{\hbox{\char`\/}}{\Wrappedafterbreak}{\hbox{\char`\/}}}%
            \catcode`\.\active
            \catcode`\,\active
            \catcode`\;\active
            \catcode`\:\active
            \catcode`\?\active
            \catcode`\!\active
            \catcode`\/\active
            \lccode`\~`\~
        }
    \makeatother

    \let\OriginalVerbatim=\Verbatim
    \makeatletter
    \renewcommand{\Verbatim}[1][1]{%
        %\parskip\z@skip
        \sbox\Wrappedcontinuationbox {\Wrappedcontinuationsymbol}%
        \sbox\Wrappedvisiblespacebox {\FV@SetupFont\Wrappedvisiblespace}%
        \def\FancyVerbFormatLine ##1{\hsize\linewidth
            \vtop{\raggedright\hyphenpenalty\z@\exhyphenpenalty\z@
                \doublehyphendemerits\z@\finalhyphendemerits\z@
                \strut ##1\strut}%
        }%
        % If the linebreak is at a space, the latter will be displayed as visible
        % space at end of first line, and a continuation symbol starts next line.
        % Stretch/shrink are however usually zero for typewriter font.
        \def\FV@Space {%
            \nobreak\hskip\z@ plus\fontdimen3\font minus\fontdimen4\font
            \discretionary{\copy\Wrappedvisiblespacebox}{\Wrappedafterbreak}
            {\kern\fontdimen2\font}%
        }%

        % Allow breaks at special characters using \PYG... macros.
        \Wrappedbreaksatspecials
        % Breaks at punctuation characters . , ; ? ! and / need catcode=\active
        \OriginalVerbatim[#1,codes*=\Wrappedbreaksatpunct]%
    }
    \makeatother

    % Exact colors from NB
    \definecolor{incolor}{HTML}{303F9F}
    \definecolor{outcolor}{HTML}{D84315}
    \definecolor{cellborder}{HTML}{CFCFCF}
    \definecolor{cellbackground}{HTML}{F7F7F7}

    % prompt
    \makeatletter
    \newcommand{\boxspacing}{\kern\kvtcb@left@rule\kern\kvtcb@boxsep}
    \makeatother
    \newcommand{\prompt}[4]{
        {\ttfamily\llap{{\color{#2}[#3]:\hspace{3pt}#4}}\vspace{-\baselineskip}}
    }
    

    
    % Prevent overflowing lines due to hard-to-break entities
    \sloppy
    % Setup hyperref package
    \hypersetup{
      breaklinks=true,  % so long urls are correctly broken across lines
      colorlinks=true,
      urlcolor=urlcolor,
      linkcolor=linkcolor,
      citecolor=citecolor,
      }
    % Slightly bigger margins than the latex defaults
    
    \geometry{verbose,tmargin=1in,bmargin=1in,lmargin=1in,rmargin=1in}
    
    

\begin{document}
    
    \maketitle
    
    

    
    \section{Estudio Mono-Compartimental farmacocinético del Sunitinib +
ketoconazol en ratones
🧪🐭}\label{estudio-mono-compartimental-farmacocinuxe9tico-del-sunitinib-ketoconazol-en-ratones}

En este cuaderno se analiza el comportamiento farmacocinético de una
formulación que contiene \textbf{sunitinib} combinado con
\textbf{ketotomasapredatidonazol} en ratones.

Utilizamos un \textbf{modelo monocompartimental con absorción de primer
orden} para describir la concentración del fármaco en distintos
compartimentos: \textbf{Plasma}, \textbf{Cerebro}, \textbf{Riñón} e
\textbf{Hígado}.

\subsubsection{Objetivos}\label{objetivos}

\begin{itemize}
\tightlist
\item
  Estimar los parámetros:

  \begin{itemize}
  \tightlist
  \item
    \(k_e\): constante de eliminación
  \item
    \(t_{1/2}\): vida media
  \item
    \(k_a\): constante de absorción
  \item
    Factor de concentración
  \end{itemize}
\item
  Calcular el error cuadrático total (SSR) y el criterio de información
  de Akaike (AIC)
\item
  Comparar los valores reales y predichos mediante gráficas
\end{itemize}

    \begin{Verbatim}[commandchars=\\\{\}]
         Cerebro    Plasma      Riñon      Higado
Tiempo
0.00    0.000000  0.000000   0.000000    0.000000
0.08    2.375455  1.253699   7.725515   28.739378
0.25    2.428088  1.883693  17.777037   43.681615
0.50    2.473455  1.635242  22.042997   53.189255
1.00    3.671161  2.028957  35.874868   80.544272
2.00    4.597385  2.850109  66.188478  158.306083
4.00    6.108580  3.384569  64.424283  119.348789
6.00    4.235917  2.608140  53.622046   90.896761
8.00    2.888443  1.869260  39.509340   71.328054
12.00   2.103540  1.188970  19.110558   27.001372
    \end{Verbatim}

    \subsection{\texorpdfstring{Función
\texttt{analizar\_compartimento}}{Función analizar\_compartimento}}\label{funciuxf3n-analizar_compartimento}

Esta función estima los parámetros cinéticos a partir de los datos
experimentales de un compartimento dado.

Se realiza el siguiente flujo:

\begin{enumerate}
\def\labelenumi{\arabic{enumi}.}
\tightlist
\item
  Se encuentra el tiempo al que se alcanza la concentración máxima
  \(C_{max}\) (es decir, \(t_{max}\)).
\item
  Se utiliza la fase terminal para ajustar una recta a ( \ln(C) ) y
  calcular:

  \begin{itemize}
  \tightlist
  \item
    \(k_e = -\text{pendiente}\)
  \item
    \(t_{1/2} = \frac{\ln(2)}{k_e}\)
  \end{itemize}
\item
  Se estima \(k_a\) utilizando el \textbf{método de bisección} con la
  ecuación:
\end{enumerate}

\[
\ln(x) = k_e \cdot (x - 1) \cdot t_{max}
\]

\begin{enumerate}
\def\labelenumi{\arabic{enumi}.}
\setcounter{enumi}{3}
\tightlist
\item
  Se ajusta el modelo monocompartimental con absorción de primer orden:
\end{enumerate}

\[
C(t) = F \cdot \left(e^{-k_e t} - e^{-k_a t} \right)
\]

Donde \(F\) es un \textbf{factor de concentración} calculado a partir de
\(C_{max}\), \(k_e\), y \(k_a\).

\begin{enumerate}
\def\labelenumi{\arabic{enumi}.}
\setcounter{enumi}{4}
\tightlist
\item
  Se calcula el \textbf{SSR} y el \textbf{AIC} para evaluar el ajuste.
\end{enumerate}

    \subsection{Aplicación del modelo a los
compartimentos}\label{aplicaciuxf3n-del-modelo-a-los-compartimentos}

Aplicamos la función \texttt{analizar\_compartimento} a los siguientes
compartimentos:

\begin{itemize}
\tightlist
\item
  Plasma
\item
  Cerebro
\item
  Riñón
\item
  Hígado
\end{itemize}

Y mostramos los parámetros estimados para cada uno.

    \subsection{Visualización de los
modelos}\label{visualizaciuxf3n-de-los-modelos}

Comparamos la concentración real con la concentración predicha por el
modelo monocompartimental con absorción para cada compartimento.

Se espera que el modelo prediga adecuadamente la fase de absorción y
eliminación.

    \begin{center}
    \adjustimage{max size={0.9\linewidth}{0.9\paperheight}}{analysis_files/analysis_7_0.png}
    \end{center}
    { \hspace*{\fill} \\}
    
    \section{Estudio Bi-Compartimental farmacocinético del Sunitinib +
ketoconazol en ratones
🧪🐭}\label{estudio-bi-compartimental-farmacocinuxe9tico-del-sunitinib-ketoconazol-en-ratones}

Este análisis se basa en un estudio farmacocinético del fármaco
\textbf{sunitinib combinado con ketotomasapredatidonazol} administrado a
ratones. El objetivo es ajustar un modelo bicompartimental con absorción
para describir la cinética del fármaco en distintos tejidos (plasma,
hígado, riñón y cerebro), y calcular parámetros como:

\begin{itemize}
\tightlist
\item
  Tasa de absorción (\texttt{ka})
\item
  Tasas de eliminación de los compartimentos (\texttt{alfa},
  \texttt{beta})
\item
  Concentración máxima (\texttt{Cmax}) y tiempo a la concentración
  máxima (\texttt{Tmax})
\item
  Área bajo la curva hasta infinito (\texttt{AUC\_inf})
\item
  Semivida terminal (\texttt{t₁/₂})
\item
  Depuración aparente (\texttt{CL/F})
\item
  Volumen de distribución aparente (\texttt{Vd/F})
\item
  Tiempo medio de residencia (\texttt{MRT})
\end{itemize}

    \begin{Verbatim}[commandchars=\\\{\}]
         Cerebro    Plasma      Riñon      Higado
Tiempo
0.00    0.000000  0.000000   0.000000    0.000000
0.08    2.375455  1.253699   7.725515   28.739378
0.25    2.428088  1.883693  17.777037   43.681615
0.50    2.473455  1.635242  22.042997   53.189255
1.00    3.671161  2.028957  35.874868   80.544272
2.00    4.597385  2.850109  66.188478  158.306083
4.00    6.108580  3.384569  64.424283  119.348789
6.00    4.235917  2.608140  53.622046   90.896761
8.00    2.888443  1.869260  39.509340   71.328054
12.00   2.103540  1.188970  19.110558   27.001372
    \end{Verbatim}

    \subsection{Modelo bicompartimental con
absorción}\label{modelo-bicompartimental-con-absorciuxf3n}

El modelo describe la concentración del fármaco \texttt{C(t)} en función
del tiempo \texttt{t} mediante la siguiente ecuación:

\[
C(t) = A \left(e^{-\alpha t} - e^{-k_a t} \right) + B \left(e^{-\beta t} - e^{-k_a t} \right)
\]

Donde: - ( k\_a ): constante de absorción - ( A ), ( B ): coeficientes
de distribución - ( \alpha ): tasa de distribución (fase rápida) - (
\beta ): tasa de eliminación terminal (fase lenta)

    \subsection{Ajuste del modelo y cálculo de parámetros
farmacocinéticos}\label{ajuste-del-modelo-y-cuxe1lculo-de-paruxe1metros-farmacocinuxe9ticos}

Se ajusta el modelo bicompartimental a los datos experimentales usando
mínimos cuadrados no lineales, y se calculan los siguientes parámetros:

\begin{itemize}
\tightlist
\item
  \textbf{Cmax}: concentración máxima predicha.
\item
  \textbf{Tmax}: tiempo en el que ocurre Cmax.
\item
  \textbf{AUC₀--∞}: área bajo la curva desde 0 hasta infinito, usando:
\end{itemize}

\[
AUC_{0-\infty} = AUC_{0-12h} + \frac{C_{\text{last}}}{\beta}
\]

\begin{itemize}
\tightlist
\item
  \textbf{t₁/₂} (semivida terminal):
\end{itemize}

\[
t_{1/2} = \frac{\ln 2}{\beta}
\]

\begin{itemize}
\tightlist
\item
  \textbf{CL/F} (depuración aparente):
\end{itemize}

\[
CL/F = \frac{D}{AUC_{0-\infty}} \quad \text{donde } D = 40 \, \mu g/kg
\]

\begin{itemize}
\tightlist
\item
  \textbf{Vd/F} (volumen de distribución aparente):
\end{itemize}

\[
Vd/F = \frac{CL/F}{\beta}
\]

\begin{itemize}
\tightlist
\item
  \textbf{MRT} (tiempo medio de residencia):
\end{itemize}

\[
MRT = \frac{Vd/F}{CL/F}
\]

\begin{itemize}
\tightlist
\item
  \textbf{SE\_\{AUC\_\{0-\infty\}\}} (error estándar del AUC
  extrapolado):
\end{itemize}

\[
SE_{AUC_{0-\infty}} = \frac{C_{\text{last}}}{\beta^2} \cdot SE_{\beta}
\]

\begin{itemize}
\tightlist
\item
  \textbf{SE\_\{CL/F\}} (error estándar de la depuración aparente, por
  propagación):
\end{itemize}

\[
SE_{CL/F} = \frac{D}{AUC_{0-\infty}^2} \cdot SE_{AUC_{0-\infty}}
\]

    \begin{Verbatim}[commandchars=\\\{\}]
            ka   alfa   beta     Cmax   Tmax   AUC\_inf  SE\_AUC\_inf   t1/2  \textbackslash{}
Tejido
Plasma   1.140  1.141  0.098    3.185  2.352    40.922      17.022  7.069
Higado   0.426  0.399  0.111  141.103  2.653  1273.524   22700.578  6.218
Riñon    0.782  0.794  0.181   68.430  3.196   641.877      85.610  3.838
Cerebro  0.800  0.800  0.130    5.550  2.714    60.887      19.471  5.349

          CL/F  SE\_CL/F   Vd/F     MRT         A        B
Tejido
Plasma   0.977    0.407  9.969  10.199     0.475    4.390
Higado   0.031    0.560  0.282   8.971  4669.783   72.466
Riñon    0.062    0.008  0.345   5.537  4349.814  169.873
Cerebro  0.657    0.210  5.070   7.717     0.753    9.416
    \end{Verbatim}

    \subsection{Comparación gráfica entre datos reales y predicción del
modelo}\label{comparaciuxf3n-gruxe1fica-entre-datos-reales-y-predicciuxf3n-del-modelo}

En esta sección se grafican las concentraciones reales observadas y las
curvas predichas por el modelo bicompartimental para cada tejido. Esto
permite visualizar qué tan bien se ajusta el modelo a los datos
experimentales.

    \begin{center}
    \adjustimage{max size={0.9\linewidth}{0.9\paperheight}}{analysis_files/analysis_17_0.png}
    \end{center}
    { \hspace*{\fill} \\}
    
    \section{Conclusiones del estudio}\label{conclusiones-del-estudio}

En esta sección final de nuestro estudio, vamos a \textbf{evaluar
cuantitativamente el ajuste de nuestros modelos} farmacocinéticos a los
datos experimentales obtenidos de ratones tratados con sunitinib. Para
ello, utilizaremos dos métricas muy importantes:

\begin{itemize}
\item
  \textbf{SSR (Suma de los Cuadrados del Residuo)}: mide el error total
  entre los valores predichos por un modelo y los valores observados.
  Cuanto menor sea el SSR, mejor es el ajuste del modelo.
\item
  \textbf{AIC (Criterio de Información de Akaike)}: evalúa la calidad de
  un modelo teniendo en cuenta tanto el ajuste a los datos como la
  complejidad del modelo (número de parámetros). Un AIC más bajo indica
  un modelo más eficiente.
\end{itemize}

Vamos a comparar \textbf{cuatro funciones modelo} y determinar cuál se
ajusta mejor a nuestros datos reales.

    \subsection{Cálculo del SSR}\label{cuxe1lculo-del-ssr}

El SSR (Suma de los Cuadrados del Residuo) se calcula usando la
siguiente fórmula:

\[
SSR = \sum_{i=1}^{n} (y_i - \hat{y}_i)^2
\]

Donde:

\begin{itemize}
\tightlist
\item
  \(y_i\) son los valores reales.
\item
  \(\hat{y}_i\) son los valores predichos por el modelo.
\end{itemize}

    \subsection{Cálculo del AIC}\label{cuxe1lculo-del-aic}

El AIC (Criterio de Información de Akaike) penaliza la complejidad del
modelo. La fórmula que usaremos es:

\[
AIC = n \cdot \ln\left(\frac{SSR}{n}\right) + 2k
\]

Donde:

\begin{itemize}
\tightlist
\item
  \(n\): número de observaciones.
\item
  \(SSR\): suma de los cuadrados de los residuos.
\item
  \(k\): número de parámetros del modelo.
\end{itemize}

Asumiremos lo siguiente para nuestros modelos: - Modelo 1: 2 parámetros
- Modelo 2: 3 parámetros - Modelo 3: 2 parámetros - Modelo 4: 4
parámetros

    \subsection{Conclusiones}\label{conclusiones}

    \begin{Verbatim}[commandchars=\\\{\}]
                     Tejido          SSR        AIC
Modelo
Bicompartimental    CEREBRO     6.324883   6.825402
Monocompartimental  CEREBRO    12.903700   9.242600
Bicompartimental     HIGADO  1485.013923  55.953594
Monocompartimental   HIGADO  4641.323000  62.209800
Bicompartimental     PLASMA     2.322398  -2.191619
Monocompartimental   PLASMA     4.276900  -0.695900
Bicompartimental      RIÑON    94.974923  31.207495
Monocompartimental    RIÑON   956.999700  47.999200
    \end{Verbatim}

    Al comparar los modelos farmacocinéticos mono y bicompartimental en
diferentes tejidos (cerebro, hígado, plasma y riñón), se observa que el
modelo bicompartimental proporciona un mejor ajuste general a los datos.
Esto se evidencia tanto por los valores más bajos de \textbf{Suma de los
Cuadrados de los Residuos (SSR)} como por los \textbf{criterios de
información de Akaike (AIC)} más reducidos en todos los tejidos
evaluados. En particular, las diferencias son más marcadas en tejidos
como el hígado y el riñón, donde el modelo bicompartimental reduce
significativamente el error de ajuste en comparación con el
monocompartimental. Estos resultados sugieren que el comportamiento
farmacocinético del compuesto estudiado se describe mejor con un enfoque
bicompartimental, reflejando posiblemente una distribución más compleja
en los tejidos analizados.

    \section{Aplicar modelo Bi-Compartimental sobre los datos de
Test}\label{aplicar-modelo-bi-compartimental-sobre-los-datos-de-test}

    \begin{Verbatim}[commandchars=\\\{\}]
         Cerebro    Plasma     Higado       Riñon
Tiempo
0.00    0.010000  0.010000   0.010000    0.010000
0.08    1.478175  0.761520   5.620105   27.757062
0.25    1.542916  1.140623  13.975576   49.453853
0.50    2.860693  2.716908  65.792679  107.025753
1.00    2.118253  1.732809  33.339625   78.384621
2.00    2.117737  1.967675  37.041934   68.008447
4.00    2.117785  1.238243  30.133649   36.716059
6.00    2.025212  1.056993  21.450501   27.188582
8.00    1.752101  0.964013  14.368102   15.055364
12.00   1.468479  0.714241   9.960894    5.880010
    \end{Verbatim}

    \begin{Verbatim}[commandchars=\\\{\}]
            ka   alfa   beta    Cmax   Tmax  AUC\_inf  SE\_AUC\_inf    t1/2  \textbackslash{}
Tejido
Plasma   2.245  2.202  0.087   2.225  0.724   22.019       8.047   7.932
Higado   2.102  1.840  0.121  45.624  0.844  359.293     144.925   5.725
Riñon    2.040  2.023  0.213  92.700  0.724  427.977      28.970   3.256
Cerebro  7.039  1.957  0.036   2.381  0.603   65.727      45.382  19.386

          CL/F  SE\_CL/F    Vd/F     MRT          A       B
Tejido
Plasma   1.817    0.664  20.788  11.443    133.456   1.898
Higado   0.111    0.045   0.920   8.260    331.606  43.277
Riñon    0.093    0.006   0.439   4.697  13162.078  89.006
Cerebro  0.609    0.420  17.020  27.967      0.368   2.357
    \end{Verbatim}

    \begin{center}
    \adjustimage{max size={0.9\linewidth}{0.9\paperheight}}{analysis_files/analysis_29_0.png}
    \end{center}
    { \hspace*{\fill} \\}
    
    \section{Análisis de diferencias de AUC usando el Test de
Yuan}\label{anuxe1lisis-de-diferencias-de-auc-usando-el-test-de-yuan}

En este análisis comparamos los valores de \textbf{AUC\_inf} entre un
grupo \textbf{control} y un grupo \textbf{inhibidor} utilizando el
\textbf{Test de Yuan}, considerando la varianza en \textbf{CL/F}
(Clearance/F).

\subsection{Funciones utilizadas}\label{funciones-utilizadas}

\subsubsection{Calcular la varianza del
AUC\_inf}\label{calcular-la-varianza-del-auc_inf}

La fórmula para la varianza de AUC es:

\[
\text{Var}(AUC) = \frac{\text{dosis}^2 \times \text{Var}(CL/F)}{(CL/F)^4}
\]

donde:

\begin{itemize}
\tightlist
\item
  \(\text{Var}(CL/F) = (SE_{CL/F})^2\)
\end{itemize}

    \subsubsection{Test de Yuan para comparar
AUC\_inf}\label{test-de-yuan-para-comparar-auc_inf}

El estadístico de prueba (Z) se calcula como:

\[
Z = \frac{AUC_{\text{control}} - AUC_{\text{inhibidor}}}{\sqrt{\frac{\text{Var}(AUC_{\text{control}})}{n_{\text{control}}} + \frac{\text{Var}(AUC_{\text{inhibidor}})}{n_{\text{inhibidor}}}}}
\]

El valor p (p-value) se calcula como:

\[
p\text{-value} = 2 \times \text{P}(Z > |z|)
\]

    \begin{Verbatim}[commandchars=\\\{\}]
         AUC\_Control  Var\_AUC\_Control  AUC\_Inhibidor  Var\_AUC\_Inhibidor  \textbackslash{}
Plasma        22.019         0.440896         40.922           0.165649
Higado       359.293         0.002025       1273.524           0.313600
Riñon        427.977         0.000036        641.877           0.000064
Cerebro       65.727         0.176400         60.887           0.044100

                    Z  p-value Significativo
Plasma     -48.543291      0.0            Sí
Higado   -3254.619643      0.0            Sí
Riñon   -42780.000000      0.0            Sí
Cerebro     20.614417      0.0            Sí
    \end{Verbatim}

    \subsection{Interpretación}\label{interpretaciuxf3n}

\begin{itemize}
\tightlist
\item
  Si el valor \textbf{p \textless{} 0.05}, se considera que hay una
  diferencia estadísticamente significativa en los valores de AUC\_inf
  entre el grupo control y el grupo inhibidor para ese tejido.
\item
  El \textbf{estadístico Z} indica cuántas desviaciones estándar separan
  los valores de AUC entre los grupos.
\end{itemize}

Se compararon los valores de AUC entre \textbf{Control} e
\textbf{Inhibidor}. Todos los cambios fueron \textbf{estadísticamente
significativos (p = 0.0)}, lo que implica un \textbf{efecto real del
inhibidor} sobre la farmacocinética en cada tejido.

\begin{longtable}[]{@{}
  >{\raggedright\arraybackslash}p{(\linewidth - 4\tabcolsep) * \real{0.2174}}
  >{\raggedright\arraybackslash}p{(\linewidth - 4\tabcolsep) * \real{0.5000}}
  >{\raggedright\arraybackslash}p{(\linewidth - 4\tabcolsep) * \real{0.2826}}@{}}
\toprule\noalign{}
\begin{minipage}[b]{\linewidth}\raggedright
Tejido
\end{minipage} & \begin{minipage}[b]{\linewidth}\raggedright
Efecto del Inhibidor
\end{minipage} & \begin{minipage}[b]{\linewidth}\raggedright
Implicación
\end{minipage} \\
\midrule\noalign{}
\endhead
\bottomrule\noalign{}
\endlastfoot
\textbf{Plasma} & ↑ AUC & Mayor exposición sistémica, el inhibidor
podría reducir el aclaramiento. \\
\textbf{Hígado} & ↑↑ AUC & Acumulación hepática marcada, posible
inhibición del metabolismo hepático. \\
\textbf{Riñón} & ↑ AUC & Mayor retención renal, podría afectar la
excreción. \\
\textbf{Cerebro} & ↓ AUC & Menor penetración al SNC, el inhibidor podría
estar bloqueando el paso a través de la barrera hematoencefálica. \\
\end{longtable}


    % Add a bibliography block to the postdoc
    
    
    
\end{document}
